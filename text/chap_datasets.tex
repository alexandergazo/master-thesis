\chapter{Datasets}
\label{chap:data}

\section{TODO}

\begin{itemize}
    \item tabulky splitov
    \item ukazka dat
    \item martinove data quality results
    \item odkaz na data na herbertovej stranke?
    \item add labeling tool screeshot
\end{itemize}

% TODO was it really mentioned
As mentioned, the primary dataset, usage-wise and inspiration-wise, was the FEVER dataset \citep{fever}.
Deriving \cite{fever} methods, we \citep{ullrich} were able to create our own FEVER-like database using Czech News Agency's infobank.

\section{FEVER}

Fact Extraction and Verification \citep{fever} dataset is a large-scale dataset based on Wikipedia articles.
The dataset was created by extracting sentences from English Wikipedia articles and classifying them by human annotators as Supported, Refuted, or NotEnoughInfo.
The evidence, either single or multiple paragraphs or even articles proving or disproving the claim, is also recorded.
The complete FEVER contains 185,445 annotated claims generated using 50,000 popular articles.

The creation consisted of two steps. The sentences were first manually extracted from popular Wikipedia articles.
Then, to create a more diverse set of claims, the annotators had the option of producing new claims by mutating the existing ones in various ways (generalizing, specification, entity substitution, non-trivial negating, and rephrasing).
More complex claims were created by providing hyperlinked Wikipedia articles as another source of information while mutating the claim.
In the second step - annotation - the annotators were asked to label the generated claims and provide suggested evidence when needed.
The whole process was streamlined in order not to spend longer than five minutes on a single claim throughout all stages.

The quality of the dataset was throughoutly tested in the paper.
One of the methods for improving the dataset was annotating the generated claim by multiple people to reduce mislabeling.

\subsection{FEVER CS}

Using translation services we created FEVER CS, which is the Czech-translated version of FEVER.
This dataset serves as a baseline to verify robustness of our models on a larger scale.

\section{\CTK}
% nomenclature

The basis for this dataset is the collection of Czech news articles provided in colaboration with Czech News Agency.
Inspired by \cite{fever} and \cite{danish_fever}, our colleague \cite{ullrich} created a czech version of the claim extracting and labeling software tool running over the ČTK infobank's articles. 
%TODO insert image
It was designed to be used by layman annotators, who were students of our partner - the Faculty of Social Sciences at Charles University.

The dataset creation consisted of two harvests. 
After reviewing the results of the first one, we were able to rewrite instructions in the tool to guide the annotators to create higher-quality mutations and labels with fewer conflicts.
The second harvest concluded with $\sim$ 3,500 labeled claims, with more than half being labeled two or more times. %TODO cite webpage
As of writing, this figure is not final as the dataset needs to be manually cleaned and have conflicts resolved. 
