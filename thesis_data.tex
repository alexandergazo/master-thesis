%%% Basic information on the thesis

\def\ThesisTitle{Algorithms for Document Retrieval in Czech Language Supporting Long Inputs}
\def\ThesisTitleSk{Metody document retrieval nad českými texty vhodné pro zpracování dlouhých vstupů}

\def\ThesisAuthor{Bc. Alexander Gažo}

\def\YearSubmitted{2021}

% Name of the department or institute where the work was officially assigned
\def\Department{Department of Computer Science}
\def\DepartmentSk{Katedra počítačů}

% Is it a department (katedra), or an institute (ústav)?
\def\DeptType{Department}
\def\DeptTypeSk{Katedra}

% Thesis supervisor: name, surname, and titles
\def\Supervisor{Ing. Jan Drchal, Ph.D.}

% Supervisor's department
%\def\SupervisorsDepartment{Department of Theoretical Computer Science, FIT}
\def\SupervisorsDepartment{Artificial Intelligence Center FEE CTU}
\def\SupervisorsDepartmentSk{Centrum Umělé Inteligence FEL ČVUT}

% Study programme and specialization
\def\StudyProgramme{Open Informatics}
\def\StudyBranch{Artificial Intelligence}

% An optional dedication
\def\Dedication{%
    TODO I would like to thank my supervisor Ing. Jan Drchal, Ph.D., for his support throughout the writing of this thesis. 
    My girlfriend Tina, who helped guide me towards the best possible outcome and last, but not least, I would like to thank my colleagues, with whom we could have endless discussions about our project.
}

% Abstract (recommended length around 80-200 words; this is not a copy of your thesis assignment!)
\def\Abstract{%
    The document retrieval task is a well-studied problem of finding the relevant subset of documents to the provided search query.
    Recent advances in the field of Natural Language Processing (NLP), namely the transformer architecture \citep{attention-is-all-you-need} and BERT model \citep{bert} provide a new approach to document retrieval.
    The document retrieval in this thesis is motivated by the Czech fact-checking task, which is an important challenge in the modern world.
    In this thesis, we apply the latest research achievements to the transformer's attention mechanism \citep{first-attention}, decreasing the space and time complexity, allowing for longer input sequences (documents TODO(vyhodit zatvorku ci ne?)).
    We then study whether the processing of whole articles, unlike only theirs paragraphs, improves the performance of the retrieval models.
}
\def\AbstractSk{%
    Úloha vyhľadávania dokumentov (document retrieval) je dobre známy problém nájdenia relevantnej podmnožiny dokumentov k vyhľadávanemu dotazu.
    Nedávny pokrok v oblasti spracovania prirodzeného jazyka (NLP), konkrétne architektúra transformera \citep{attention-is-all-you-need} a model BERT \citep{bert}, poskytujú nový prístup k vyhľadávaniu dokumentov.
    Vyhľadávanie dokumentov v tejto práci je motivované úlohou overovania faktov v českom jazyku, ktorá je dôležitou výzvou pre moderný svet.
    V tejto práci aplikujeme najnovšie výskumné výsledky na mechanizmus pozornosti (attention) transformera \citep{first-attention}, znižujúc priestorovú a časovú zložitosť, čo umožňuje prácu s dlhšími vstupnými sekvenciami (dokumentami TODO).
    Na záver skúmame, či spracovanie celých článkov, na rozdiel od iba ich odsekov, zlepšuje výkonnosť vyhľadávacích modelov.
}

% 3 to 5 keywords (recommended), each enclosed in curly braces
\def\Keywords{%
    {document retrieval},
    {fact-checking},
    {long-inputs},
    {Czech language},
    {NLP},
    {BERT},
    {TFIDF}
}
\def\KeywordsSk{%
    {vyhľadávanie dokumentov},
    {overovanie faktov},
    {dlhé vstupy},
    {český jazyk},
    {NLP},
    {BERT},
    {TFIDF}
}
